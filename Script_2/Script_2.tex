% Options for packages loaded elsewhere
\PassOptionsToPackage{unicode}{hyperref}
\PassOptionsToPackage{hyphens}{url}
%
\documentclass[
]{article}
\usepackage{amsmath,amssymb}
\usepackage{iftex}
\ifPDFTeX
  \usepackage[T1]{fontenc}
  \usepackage[utf8]{inputenc}
  \usepackage{textcomp} % provide euro and other symbols
\else % if luatex or xetex
  \usepackage{unicode-math} % this also loads fontspec
  \defaultfontfeatures{Scale=MatchLowercase}
  \defaultfontfeatures[\rmfamily]{Ligatures=TeX,Scale=1}
\fi
\usepackage{lmodern}
\ifPDFTeX\else
  % xetex/luatex font selection
\fi
% Use upquote if available, for straight quotes in verbatim environments
\IfFileExists{upquote.sty}{\usepackage{upquote}}{}
\IfFileExists{microtype.sty}{% use microtype if available
  \usepackage[]{microtype}
  \UseMicrotypeSet[protrusion]{basicmath} % disable protrusion for tt fonts
}{}
\makeatletter
\@ifundefined{KOMAClassName}{% if non-KOMA class
  \IfFileExists{parskip.sty}{%
    \usepackage{parskip}
  }{% else
    \setlength{\parindent}{0pt}
    \setlength{\parskip}{6pt plus 2pt minus 1pt}}
}{% if KOMA class
  \KOMAoptions{parskip=half}}
\makeatother
\usepackage{xcolor}
\usepackage[margin=1in]{geometry}
\usepackage{color}
\usepackage{fancyvrb}
\newcommand{\VerbBar}{|}
\newcommand{\VERB}{\Verb[commandchars=\\\{\}]}
\DefineVerbatimEnvironment{Highlighting}{Verbatim}{commandchars=\\\{\}}
% Add ',fontsize=\small' for more characters per line
\usepackage{framed}
\definecolor{shadecolor}{RGB}{248,248,248}
\newenvironment{Shaded}{\begin{snugshade}}{\end{snugshade}}
\newcommand{\AlertTok}[1]{\textcolor[rgb]{0.94,0.16,0.16}{#1}}
\newcommand{\AnnotationTok}[1]{\textcolor[rgb]{0.56,0.35,0.01}{\textbf{\textit{#1}}}}
\newcommand{\AttributeTok}[1]{\textcolor[rgb]{0.13,0.29,0.53}{#1}}
\newcommand{\BaseNTok}[1]{\textcolor[rgb]{0.00,0.00,0.81}{#1}}
\newcommand{\BuiltInTok}[1]{#1}
\newcommand{\CharTok}[1]{\textcolor[rgb]{0.31,0.60,0.02}{#1}}
\newcommand{\CommentTok}[1]{\textcolor[rgb]{0.56,0.35,0.01}{\textit{#1}}}
\newcommand{\CommentVarTok}[1]{\textcolor[rgb]{0.56,0.35,0.01}{\textbf{\textit{#1}}}}
\newcommand{\ConstantTok}[1]{\textcolor[rgb]{0.56,0.35,0.01}{#1}}
\newcommand{\ControlFlowTok}[1]{\textcolor[rgb]{0.13,0.29,0.53}{\textbf{#1}}}
\newcommand{\DataTypeTok}[1]{\textcolor[rgb]{0.13,0.29,0.53}{#1}}
\newcommand{\DecValTok}[1]{\textcolor[rgb]{0.00,0.00,0.81}{#1}}
\newcommand{\DocumentationTok}[1]{\textcolor[rgb]{0.56,0.35,0.01}{\textbf{\textit{#1}}}}
\newcommand{\ErrorTok}[1]{\textcolor[rgb]{0.64,0.00,0.00}{\textbf{#1}}}
\newcommand{\ExtensionTok}[1]{#1}
\newcommand{\FloatTok}[1]{\textcolor[rgb]{0.00,0.00,0.81}{#1}}
\newcommand{\FunctionTok}[1]{\textcolor[rgb]{0.13,0.29,0.53}{\textbf{#1}}}
\newcommand{\ImportTok}[1]{#1}
\newcommand{\InformationTok}[1]{\textcolor[rgb]{0.56,0.35,0.01}{\textbf{\textit{#1}}}}
\newcommand{\KeywordTok}[1]{\textcolor[rgb]{0.13,0.29,0.53}{\textbf{#1}}}
\newcommand{\NormalTok}[1]{#1}
\newcommand{\OperatorTok}[1]{\textcolor[rgb]{0.81,0.36,0.00}{\textbf{#1}}}
\newcommand{\OtherTok}[1]{\textcolor[rgb]{0.56,0.35,0.01}{#1}}
\newcommand{\PreprocessorTok}[1]{\textcolor[rgb]{0.56,0.35,0.01}{\textit{#1}}}
\newcommand{\RegionMarkerTok}[1]{#1}
\newcommand{\SpecialCharTok}[1]{\textcolor[rgb]{0.81,0.36,0.00}{\textbf{#1}}}
\newcommand{\SpecialStringTok}[1]{\textcolor[rgb]{0.31,0.60,0.02}{#1}}
\newcommand{\StringTok}[1]{\textcolor[rgb]{0.31,0.60,0.02}{#1}}
\newcommand{\VariableTok}[1]{\textcolor[rgb]{0.00,0.00,0.00}{#1}}
\newcommand{\VerbatimStringTok}[1]{\textcolor[rgb]{0.31,0.60,0.02}{#1}}
\newcommand{\WarningTok}[1]{\textcolor[rgb]{0.56,0.35,0.01}{\textbf{\textit{#1}}}}
\usepackage{graphicx}
\makeatletter
\def\maxwidth{\ifdim\Gin@nat@width>\linewidth\linewidth\else\Gin@nat@width\fi}
\def\maxheight{\ifdim\Gin@nat@height>\textheight\textheight\else\Gin@nat@height\fi}
\makeatother
% Scale images if necessary, so that they will not overflow the page
% margins by default, and it is still possible to overwrite the defaults
% using explicit options in \includegraphics[width, height, ...]{}
\setkeys{Gin}{width=\maxwidth,height=\maxheight,keepaspectratio}
% Set default figure placement to htbp
\makeatletter
\def\fps@figure{htbp}
\makeatother
\setlength{\emergencystretch}{3em} % prevent overfull lines
\providecommand{\tightlist}{%
  \setlength{\itemsep}{0pt}\setlength{\parskip}{0pt}}
\setcounter{secnumdepth}{-\maxdimen} % remove section numbering
\ifLuaTeX
  \usepackage{selnolig}  % disable illegal ligatures
\fi
\usepackage{bookmark}
\IfFileExists{xurl.sty}{\usepackage{xurl}}{} % add URL line breaks if available
\urlstyle{same}
\hypersetup{
  pdftitle={Filtrando dados},
  pdfauthor={Miguel Carvalho Nascimento},
  hidelinks,
  pdfcreator={LaTeX via pandoc}}

\title{Filtrando dados}
\author{Miguel Carvalho Nascimento}
\date{}

\begin{document}
\maketitle

\subsection{Comparação de
Desempenho}\label{comparauxe7uxe3o-de-desempenho}

Carregando pacotes necessários:

\begin{Shaded}
\begin{Highlighting}[]
\DocumentationTok{\#\#\#\#\#\#\# Nao aparecer na saida do html}
\CommentTok{\# Carregando pacotes}
\FunctionTok{library}\NormalTok{(dplyr)}
\end{Highlighting}
\end{Shaded}

\begin{verbatim}

Anexando pacote: 'dplyr'
\end{verbatim}

\begin{verbatim}
Os seguintes objetos são mascarados por 'package:stats':

    filter, lag
\end{verbatim}

\begin{verbatim}
Os seguintes objetos são mascarados por 'package:base':

    intersect, setdiff, setequal, union
\end{verbatim}

\begin{Shaded}
\begin{Highlighting}[]
\FunctionTok{library}\NormalTok{(data.table)}
\end{Highlighting}
\end{Shaded}

\begin{verbatim}

Anexando pacote: 'data.table'
\end{verbatim}

\begin{verbatim}
Os seguintes objetos são mascarados por 'package:dplyr':

    between, first, last
\end{verbatim}

\begin{Shaded}
\begin{Highlighting}[]
\FunctionTok{library}\NormalTok{(microbenchmark)}
\end{Highlighting}
\end{Shaded}

\paragraph{Criando objeto para testes de
desempenho:}\label{criando-objeto-para-testes-de-desempenho}

\begin{Shaded}
\begin{Highlighting}[]
\CommentTok{\# Dados simulados}
\NormalTok{df }\OtherTok{\textless{}{-}} \FunctionTok{data.frame}\NormalTok{(}\AttributeTok{a =} \FunctionTok{rnorm}\NormalTok{(}\FloatTok{1e6}\NormalTok{), }\AttributeTok{b =} \FunctionTok{rnorm}\NormalTok{(}\FloatTok{1e6}\NormalTok{))}
\FunctionTok{dim}\NormalTok{(df)}
\end{Highlighting}
\end{Shaded}

\begin{verbatim}
[1] 1000000       2
\end{verbatim}

\paragraph{Exemplo com ``r-base''}\label{exemplo-com-r-base}

\begin{Shaded}
\begin{Highlighting}[]
\CommentTok{\# Método clássico com base R}
\FunctionTok{system.time}\NormalTok{(\{}
\NormalTok{  res\_base }\OtherTok{\textless{}{-}}\NormalTok{ df[df}\SpecialCharTok{$}\NormalTok{a }\SpecialCharTok{\textgreater{}} \DecValTok{0} \SpecialCharTok{\&}\NormalTok{ df}\SpecialCharTok{$}\NormalTok{b }\SpecialCharTok{\textgreater{}} \DecValTok{0}\NormalTok{, ]}
\NormalTok{\})}
\end{Highlighting}
\end{Shaded}

\begin{verbatim}
##   usuário   sistema decorrido 
##     0.016     0.012     0.027
\end{verbatim}

\paragraph{Exemplo com ``dplyr''}\label{exemplo-com-dplyr}

\begin{Shaded}
\begin{Highlighting}[]
\CommentTok{\# Método otimizado com dplyr}
\FunctionTok{system.time}\NormalTok{(\{}
\NormalTok{  res\_dplyr }\OtherTok{\textless{}{-}}\NormalTok{ df }\SpecialCharTok{\%\textgreater{}\%}
    \FunctionTok{filter}\NormalTok{(a }\SpecialCharTok{\textgreater{}} \DecValTok{0}\NormalTok{, b }\SpecialCharTok{\textgreater{}} \DecValTok{0}\NormalTok{)}
\NormalTok{\})}
\end{Highlighting}
\end{Shaded}

\begin{verbatim}
  usuário   sistema decorrido 
    0.017     0.005     0.021 
\end{verbatim}

\paragraph{Exemplo com ``data.table''}\label{exemplo-com-data.table}

\begin{Shaded}
\begin{Highlighting}[]
\CommentTok{\# Método otimizado com data.table}
\NormalTok{dt }\OtherTok{\textless{}{-}} \FunctionTok{as.data.table}\NormalTok{(df)}
\FunctionTok{system.time}\NormalTok{(\{}
\NormalTok{  res\_dt }\OtherTok{\textless{}{-}}\NormalTok{ dt[a }\SpecialCharTok{\textgreater{}} \DecValTok{0} \SpecialCharTok{\&}\NormalTok{ b }\SpecialCharTok{\textgreater{}} \DecValTok{0}\NormalTok{]}
\NormalTok{\})}
\end{Highlighting}
\end{Shaded}

\begin{verbatim}
  usuário   sistema decorrido 
    0.019     0.000     0.017 
\end{verbatim}

\paragraph{Comparação}\label{comparauxe7uxe3o}

\begin{Shaded}
\begin{Highlighting}[]
\FunctionTok{microbenchmark}\NormalTok{(}
  \AttributeTok{base =}\NormalTok{ df[df}\SpecialCharTok{$}\NormalTok{a }\SpecialCharTok{\textgreater{}} \DecValTok{0} \SpecialCharTok{\&}\NormalTok{ df}\SpecialCharTok{$}\NormalTok{b }\SpecialCharTok{\textgreater{}} \DecValTok{0}\NormalTok{, ],}
  \AttributeTok{dplyr =}\NormalTok{ df }\SpecialCharTok{\%\textgreater{}\%} \FunctionTok{filter}\NormalTok{(a }\SpecialCharTok{\textgreater{}} \DecValTok{0}\NormalTok{, b }\SpecialCharTok{\textgreater{}} \DecValTok{0}\NormalTok{),}
  \AttributeTok{data\_table =}\NormalTok{ dt[a }\SpecialCharTok{\textgreater{}} \DecValTok{0} \SpecialCharTok{\&}\NormalTok{ b }\SpecialCharTok{\textgreater{}} \DecValTok{0}\NormalTok{]}
\NormalTok{)}
\end{Highlighting}
\end{Shaded}

\begin{verbatim}
Unit: milliseconds
       expr      min       lq     mean   median       uq      max neval cld
       base 20.63006 22.49679 25.87129 23.72542 26.04441 69.55099   100  a 
      dplyr 12.50905 14.58043 16.44222 15.37457 16.92572 51.98931   100   b
 data_table 12.24481 12.77531 14.57271 14.03327 15.39062 48.38506   100   b
\end{verbatim}

\paragraph{Colunas Explicadas}\label{colunas-explicadas}

\begin{itemize}
\item
  \textbf{Unit:} microseconds: A unidade de tempo usada para medir as
  expressões é microsegundos (1 microsegundo = 10\^{}-6 segundos).
\item
  \textbf{expr:} A expressão sendo avaliada.
\item
  \textbf{min:} O menor tempo de execução observado entre todas as
  execuções.
\item
  \textbf{lq (lower quartile):} O valor do primeiro quartil, ou seja,
  25\% das execuções foram concluídas em menos tempo que este valor.
\item
  \textbf{mean:} A média dos tempos de execução.
\item
  \textbf{median:} O valor mediano dos tempos de execução, ou seja, 50\%
  das execuções foram concluídas em menos tempo que este valor.
\item
  \textbf{uq (upper quartile):} O valor do terceiro quartil, ou seja,
  75\% das execuções foram concluídas em menos tempo que este valor.
\item
  \textbf{max:} O maior tempo de execução observado entre todas as
  execuções.
\item
  \textbf{neval:} O número de execuções realizadas para cada expressão.
\end{itemize}

\end{document}
