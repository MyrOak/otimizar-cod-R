% Options for packages loaded elsewhere
\PassOptionsToPackage{unicode}{hyperref}
\PassOptionsToPackage{hyphens}{url}
%
\documentclass[
]{article}
\usepackage{amsmath,amssymb}
\usepackage{iftex}
\ifPDFTeX
  \usepackage[T1]{fontenc}
  \usepackage[utf8]{inputenc}
  \usepackage{textcomp} % provide euro and other symbols
\else % if luatex or xetex
  \usepackage{unicode-math} % this also loads fontspec
  \defaultfontfeatures{Scale=MatchLowercase}
  \defaultfontfeatures[\rmfamily]{Ligatures=TeX,Scale=1}
\fi
\usepackage{lmodern}
\ifPDFTeX\else
  % xetex/luatex font selection
\fi
% Use upquote if available, for straight quotes in verbatim environments
\IfFileExists{upquote.sty}{\usepackage{upquote}}{}
\IfFileExists{microtype.sty}{% use microtype if available
  \usepackage[]{microtype}
  \UseMicrotypeSet[protrusion]{basicmath} % disable protrusion for tt fonts
}{}
\makeatletter
\@ifundefined{KOMAClassName}{% if non-KOMA class
  \IfFileExists{parskip.sty}{%
    \usepackage{parskip}
  }{% else
    \setlength{\parindent}{0pt}
    \setlength{\parskip}{6pt plus 2pt minus 1pt}}
}{% if KOMA class
  \KOMAoptions{parskip=half}}
\makeatother
\usepackage{xcolor}
\usepackage[margin=1in]{geometry}
\usepackage{color}
\usepackage{fancyvrb}
\newcommand{\VerbBar}{|}
\newcommand{\VERB}{\Verb[commandchars=\\\{\}]}
\DefineVerbatimEnvironment{Highlighting}{Verbatim}{commandchars=\\\{\}}
% Add ',fontsize=\small' for more characters per line
\usepackage{framed}
\definecolor{shadecolor}{RGB}{248,248,248}
\newenvironment{Shaded}{\begin{snugshade}}{\end{snugshade}}
\newcommand{\AlertTok}[1]{\textcolor[rgb]{0.94,0.16,0.16}{#1}}
\newcommand{\AnnotationTok}[1]{\textcolor[rgb]{0.56,0.35,0.01}{\textbf{\textit{#1}}}}
\newcommand{\AttributeTok}[1]{\textcolor[rgb]{0.13,0.29,0.53}{#1}}
\newcommand{\BaseNTok}[1]{\textcolor[rgb]{0.00,0.00,0.81}{#1}}
\newcommand{\BuiltInTok}[1]{#1}
\newcommand{\CharTok}[1]{\textcolor[rgb]{0.31,0.60,0.02}{#1}}
\newcommand{\CommentTok}[1]{\textcolor[rgb]{0.56,0.35,0.01}{\textit{#1}}}
\newcommand{\CommentVarTok}[1]{\textcolor[rgb]{0.56,0.35,0.01}{\textbf{\textit{#1}}}}
\newcommand{\ConstantTok}[1]{\textcolor[rgb]{0.56,0.35,0.01}{#1}}
\newcommand{\ControlFlowTok}[1]{\textcolor[rgb]{0.13,0.29,0.53}{\textbf{#1}}}
\newcommand{\DataTypeTok}[1]{\textcolor[rgb]{0.13,0.29,0.53}{#1}}
\newcommand{\DecValTok}[1]{\textcolor[rgb]{0.00,0.00,0.81}{#1}}
\newcommand{\DocumentationTok}[1]{\textcolor[rgb]{0.56,0.35,0.01}{\textbf{\textit{#1}}}}
\newcommand{\ErrorTok}[1]{\textcolor[rgb]{0.64,0.00,0.00}{\textbf{#1}}}
\newcommand{\ExtensionTok}[1]{#1}
\newcommand{\FloatTok}[1]{\textcolor[rgb]{0.00,0.00,0.81}{#1}}
\newcommand{\FunctionTok}[1]{\textcolor[rgb]{0.13,0.29,0.53}{\textbf{#1}}}
\newcommand{\ImportTok}[1]{#1}
\newcommand{\InformationTok}[1]{\textcolor[rgb]{0.56,0.35,0.01}{\textbf{\textit{#1}}}}
\newcommand{\KeywordTok}[1]{\textcolor[rgb]{0.13,0.29,0.53}{\textbf{#1}}}
\newcommand{\NormalTok}[1]{#1}
\newcommand{\OperatorTok}[1]{\textcolor[rgb]{0.81,0.36,0.00}{\textbf{#1}}}
\newcommand{\OtherTok}[1]{\textcolor[rgb]{0.56,0.35,0.01}{#1}}
\newcommand{\PreprocessorTok}[1]{\textcolor[rgb]{0.56,0.35,0.01}{\textit{#1}}}
\newcommand{\RegionMarkerTok}[1]{#1}
\newcommand{\SpecialCharTok}[1]{\textcolor[rgb]{0.81,0.36,0.00}{\textbf{#1}}}
\newcommand{\SpecialStringTok}[1]{\textcolor[rgb]{0.31,0.60,0.02}{#1}}
\newcommand{\StringTok}[1]{\textcolor[rgb]{0.31,0.60,0.02}{#1}}
\newcommand{\VariableTok}[1]{\textcolor[rgb]{0.00,0.00,0.00}{#1}}
\newcommand{\VerbatimStringTok}[1]{\textcolor[rgb]{0.31,0.60,0.02}{#1}}
\newcommand{\WarningTok}[1]{\textcolor[rgb]{0.56,0.35,0.01}{\textbf{\textit{#1}}}}
\usepackage{graphicx}
\makeatletter
\def\maxwidth{\ifdim\Gin@nat@width>\linewidth\linewidth\else\Gin@nat@width\fi}
\def\maxheight{\ifdim\Gin@nat@height>\textheight\textheight\else\Gin@nat@height\fi}
\makeatother
% Scale images if necessary, so that they will not overflow the page
% margins by default, and it is still possible to overwrite the defaults
% using explicit options in \includegraphics[width, height, ...]{}
\setkeys{Gin}{width=\maxwidth,height=\maxheight,keepaspectratio}
% Set default figure placement to htbp
\makeatletter
\def\fps@figure{htbp}
\makeatother
\setlength{\emergencystretch}{3em} % prevent overfull lines
\providecommand{\tightlist}{%
  \setlength{\itemsep}{0pt}\setlength{\parskip}{0pt}}
\setcounter{secnumdepth}{-\maxdimen} % remove section numbering
\ifLuaTeX
  \usepackage{selnolig}  % disable illegal ligatures
\fi
\usepackage{bookmark}
\IfFileExists{xurl.sty}{\usepackage{xurl}}{} % add URL line breaks if available
\urlstyle{same}
\hypersetup{
  pdftitle={Apply vs For},
  pdfauthor={Miguel Carvalho Nascimento},
  hidelinks,
  pdfcreator={LaTeX via pandoc}}

\title{Apply vs For}
\author{Miguel Carvalho Nascimento}
\date{}

\begin{document}
\maketitle

\subsection{Comparação de
Desempenho}\label{comparauxe7uxe3o-de-desempenho}

Usando o pacote microbenchmark para comparar:

\begin{Shaded}
\begin{Highlighting}[]
\CommentTok{\#install.packages("microbenchmark")}
\FunctionTok{library}\NormalTok{(microbenchmark)}
\end{Highlighting}
\end{Shaded}

\begin{verbatim}
## Warning: package 'microbenchmark' was built under R version 4.3.3
\end{verbatim}

\paragraph{Criando objeto para testes de
desempenho:}\label{criando-objeto-para-testes-de-desempenho}

\begin{Shaded}
\begin{Highlighting}[]
\NormalTok{(mat }\OtherTok{\textless{}{-}} \FunctionTok{matrix}\NormalTok{(}\DecValTok{1}\SpecialCharTok{:}\DecValTok{9}\NormalTok{, }\AttributeTok{nrow =} \DecValTok{3}\NormalTok{))}
\end{Highlighting}
\end{Shaded}

\begin{verbatim}
##      [,1] [,2] [,3]
## [1,]    1    4    7
## [2,]    2    5    8
## [3,]    3    6    9
\end{verbatim}

\paragraph{Exemplo com apply}\label{exemplo-com-apply}

\begin{Shaded}
\begin{Highlighting}[]
\NormalTok{(result\_apply }\OtherTok{\textless{}{-}} \FunctionTok{apply}\NormalTok{(mat, }\DecValTok{1}\NormalTok{, sum))}
\end{Highlighting}
\end{Shaded}

\begin{verbatim}
## [1] 12 15 18
\end{verbatim}

\paragraph{Exemplo com for}\label{exemplo-com-for}

\begin{Shaded}
\begin{Highlighting}[]
\NormalTok{result\_for }\OtherTok{\textless{}{-}} \FunctionTok{numeric}\NormalTok{(}\FunctionTok{nrow}\NormalTok{(mat))}
\ControlFlowTok{for}\NormalTok{ (i }\ControlFlowTok{in} \DecValTok{1}\SpecialCharTok{:}\FunctionTok{nrow}\NormalTok{(mat)) \{}
\NormalTok{  result\_for[i] }\OtherTok{\textless{}{-}} \FunctionTok{sum}\NormalTok{(mat[i, ])}
\NormalTok{\};result\_for}
\end{Highlighting}
\end{Shaded}

\begin{verbatim}
## [1] 12 15 18
\end{verbatim}

\paragraph{Comparação}\label{comparauxe7uxe3o}

\begin{Shaded}
\begin{Highlighting}[]
\FunctionTok{microbenchmark}\NormalTok{(}
  \AttributeTok{apply =} \FunctionTok{apply}\NormalTok{(mat, }\DecValTok{1}\NormalTok{, sum),}
  \AttributeTok{for\_loop =}\NormalTok{ \{}
\NormalTok{    result\_for }\OtherTok{\textless{}{-}} \FunctionTok{numeric}\NormalTok{(}\FunctionTok{nrow}\NormalTok{(mat))}
    \ControlFlowTok{for}\NormalTok{ (i }\ControlFlowTok{in} \DecValTok{1}\SpecialCharTok{:}\FunctionTok{nrow}\NormalTok{(mat)) \{}
\NormalTok{      result\_for[i] }\OtherTok{\textless{}{-}} \FunctionTok{sum}\NormalTok{(mat[i, ])}
\NormalTok{    \}}
\NormalTok{  \}}
\NormalTok{)}
\end{Highlighting}
\end{Shaded}

\begin{verbatim}
## Unit: microseconds
##      expr    min      lq     mean  median      uq    max neval
##     apply   17.8   22.10   31.140   30.55   34.15  128.3   100
##  for_loop 1917.9 1988.75 2251.888 2058.90 2453.80 3872.5   100
\end{verbatim}

\paragraph{Colunas Explicadas}\label{colunas-explicadas}

\begin{itemize}
\item
  \textbf{Unit:} microseconds: A unidade de tempo usada para medir as
  expressões é microsegundos (1 microsegundo = 10\^{}-6 segundos).
\item
  \textbf{expr:} A expressão sendo avaliada. No caso, temos apply e
  for\_loop.
\item
  \textbf{min:} O menor tempo de execução observado entre todas as
  execuções. apply tem um mínimo de 18.6 microsegundos, enquanto
  for\_loop tem um mínimo de 1975.2 microsegundos.
\item
  \textbf{lq (lower quartile):} O valor do primeiro quartil, ou seja,
  25\% das execuções foram concluídas em menos tempo que este valor.
  apply tem um lq de 23.90 microsegundos, e for\_loop tem um lq de
  2036.85 microsegundos.
\item
  \textbf{mean:} A média dos tempos de execução. apply tem um tempo
  médio de 30.746 microsegundos, enquanto for\_loop tem uma média de
  2277.950 microsegundos.
\item
  \textbf{median:} O valor mediano dos tempos de execução, ou seja, 50\%
  das execuções foram concluídas em menos tempo que este valor. apply
  tem uma mediana de 31.25 microsegundos e for\_loop tem uma mediana de
  2082.95 microsegundos.
\item
  \textbf{uq (upper quartile):} O valor do terceiro quartil, ou seja,
  75\% das execuções foram concluídas em menos tempo que este valor.
  apply tem um uq de 35.15 microsegundos, e for\_loop tem um uq de
  2220.15 microsegundos.
\item
  \textbf{max:} O maior tempo de execução observado entre todas as
  execuções. apply tem um máximo de 93.7 microsegundos, enquanto
  for\_loop tem um máximo de 5675.7 microsegundos.
\item
  \textbf{neval:} O número de execuções realizadas para cada expressão.
  Ambas as expressões foram executadas 100 vezes.
\end{itemize}

\end{document}
